
\documentclass[a4paper,12pt]{report}

\input{os-design.sty}

% \newcommand{\OSName}{\textit{Nachos}}
\newcommand{\OSName}{\textit{Pintos}}

\title{Design-ul Sistemului de Operare \OSName{}}

\author{Butiri Dan, Lazar Laurentiu, Pârvu Leonard}

\onehalfspacing

\begin{document}

\selectlanguage{english}
%\selectlanguage{romanian}


\maketitle

\chapter{Prezentare generală}

\section{Echipa}

\begin{enumerate}
	\item Butiri Dan
	    \begin{enumerate}
	     \item Threads: Priority Scheduler
	    \end{enumerate}

	\item Lazar Laurentiu
	    \begin{enumerate}
	     \item Threads: Priority Scheduler
	    \end{enumerate}
	    
	\item Pârvu Leonard
	    \begin{enumerate}
	     \item Threads: Alarm Clock
	     \item Threads: Multilevel Feedback Queue Scheduler
	    \end{enumerate}
\end{enumerate}


\chapter{Design-ul Modulului \textit{Threads}}

\section{Alarm Clock}

  Aceasta prima parte a proiectului consta in reimplementarea functiei \textit{timer\_sleep()} din
\textit{"devices/timer.c"}. Implementarea initiala presupune ocuparea procesorului atunci cand se apeleaza 
aceasta functie pentru un thread, el fiind pastrat in \textit{ready\_list}.

\subsection{Structuri de date}
Pentru a evita acest busy waiting am ales sa adaugam structurii thread din \textit{"threads/thread.h"} un field
\textit{wakeup\_time} pentru a memora timpul cand trebuie sa se trezeasca threadul respectiv.

De asemenea, in \textit{"devices/timer.c"}, se va adauga o noua lista, \textit{sleeping\_list}, ce contine toate threaduri
pentru care a fost apelata functia \textit{timer\_sleep()}.

Astfel atunci cand pentru un thread se apeleaza functia \textit{timer\_sleep()}, acesta va fi mutat in 
\textit{sleeping\_list} in loc sa fie pastrat in \textit{ready\_list}, situatie in care tine procesorul ocupat.

\begin{lstlisting}
struct thread
  {
    /* Owned by thread.c. */
    tid_t tid;                          /* Thread identifier. */
    enum thread_status status;          /* Thread state. */
    char name[16];                      /* Name (for debugging purposes). */
    uint8_t *stack;                     /* Saved stack pointer. */
    int priority;                       /* Priority. */
    struct list_elem allelem;           /* List element for all threads list. */

		int64_t wakeup_time;

    /* Shared between thread.c and synch.c. */
    struct list_elem elem;              /* List element. */

#ifdef USERPROG
    /* Owned by userprog/process.c. */
    uint32_t *pagedir;                  /* Page directory. */
#endif

    /* Owned by thread.c. */
    unsigned magic;                     /* Detects stack overflow. */
  };
\end{lstlisting}

\subsection{Algoritmi}
  Atunci cand se va apela functia \textit{timer\_sleep()}, se va lua threadul curent, se va calcula si memora timpul
la care trebuie sa se trezeasca, timp\_curent + sleep\_time, se va insera in lista \textit{sleeping\_list} si se va 
scoate din lista \textit{ready\_list}.

La fiecare apel al functie \textit{timer\_interrupt()} se va parcurge lista \textit{sleeping\_list} pentru a vedea daca
exista un thread ce trebuie trezit si introdus in lista de threaduri {ready\_list}.

Pentru a minimiza timpul petrecut pentru a parcurge lista {sleeping\_list}, in cadrul metodei 
\textit{timer\_interrupt()} se vor pastra threadurile sortate crescator dupa timpul la care trebuie sa se
trezeasca. Astfel, nu trebuie parcursa intreaga lista. Se va parcurge lista doar pana cand intalnim un 
thread care trebuie trezit, dupa acela sigur nefiind alt thread ce trebuie trezit. In cazul in care 
nu trebuie trezit niciun thread, lista se va parcurge in totalitate.
\subsection{Sincronizare}
Pentru a evita conditiile de cursa atunci cand mai multe threaduri apeleaza \textit{timer\_sleep()}, se vor 
dezactiva intreruperile temporar, pana la terminarea functiei.

Astfel, nu pot aparea intreruperi atunci cand se efectueaza un apel la functia \textit{timer\_sleep()}. Nu exista
conditii de cursa.

\section{Priority Scheduler}

\subsection{Structuri de date}
Deoarece prioritatea unui thread se poate schimba, am ales sa adaugam un nou camp intreg \textit{initial\_priority},
in structura thread, pentru a retine prioritatea unui thread inainte de a primii donatii. 

Tot in structura thread, am adaugat un pointer catre o structura de tip lock, in care se va retine lacatul 
dupa care asteapta threadul curent. Acest lacat, avand un \textit{lock holder}, mai ajuta thread-ul curent si 
prin faptul ca retine thread-ul care a primit donatie de prioritate de la thread-ul curent (un thread poate, 
la un moment dat, sa astepte dupa un singur lacat, si deci, sa ofere donatie directa unui singur thread).

Deasemenea, un thread a mai primit si o lista de lacate pe care le-a obtinut, aceasta lista fiind necesara la
recalcularea prioritatilor dupa eliberarea unui lacat.
\begin{lstlisting}
struct thread
  {
    /* Owned by thread.c. */
    tid_t tid;                          /* Thread identifier. */
    enum thread_status status;          /* Thread state. */
    char name[16];                      /* Name (for debugging purposes). */
    uint8_t *stack;                     /* Saved stack pointer. */
    int priority;                       /* Priority. */
    struct list_elem allelem;           /* List element for all threads list. */

    /* Shared between thread.c and synch.c. */
    struct list_elem elem;              /* List element. */
     
    /* Added for Priority Scheduler */
    int initial_priority;               /* Thread's initial priority, 
					   not altered by donations. */
    struct list acquired_locks;
    struct lock* waiting_on_lock;       /* The lock for which the thread 
					   was blocked. */

  #ifdef USERPROG
    /* Owned by userprog/process.c. */
    uint32_t *pagedir;                  /* Page directory. */
  #endif

    /* Owned by thread.c. */
    unsigned magic;                     /* Detects stack overflow. */
  };
\end{lstlisting}

Strucutra de lacat s-a schimbat prin faptul ca lacatele sunt implementate fara ajutorul semafoarelor, deci campurile
dintr-un semafor apar acum si in structura de lacat. Deasemenea, in structura de lacat a fost nevoie de un
element pentru liste, cu ajutorul caruia un lacat este adaugat in lista de \textit{acquired\_locks} a unui thread.

\begin{lstlisting}
struct lock 
  {
    struct thread *holder; /* Thread holding lock (for debugging). */

    /* Added for Priority Scheduler */
    bool is_open;          /* If is_open == true, the 
		              lock can be acquired. */
    struct list waiters;

    struct list_elem elem; /* List element for a thread 
		              acquired_locks list */
  };
\end{lstlisting}


\subsection{Algoritmi}
Pentru a planifica intotdeauna thead-ul cu prioritatea cea mai mare, thread-urile sunt pastrate in \textit{ready\_list}
in ordinea prioritatilor. La inserarea unui thread cu o prioritatea care mai exista in \textit{ready\_list}, acesta va
fi pus in spatele tuturor thread-urilor cu care este egal ca prioritate. Intotdeauna va fi planificat primul thread din 
\textit{ready\_list}.

Preemptivitatea este asigurata prin faptul ca, de fiecare data cand apare un nou thread in sistem sau de fiecare data 
cand prioritatea unui thread este modificata, se verifica daca exista in \textit{ready\_list} un thread cu prioritate
mai mare decat cea a thread-ului curent. Daca da, atunci thread-ul curent este suspendat si se apeleaza planificatorul.

Pentru problemele de sincronizare cu lacate, algoritmul este impartit in doua locuri, in functiile \textit{lock\_acquire}
si \textit{lock\_release}.

La apelul functiei \textit{lock\_acquire}, in cazul in care lacatul este ocupat, thread-ul curent va fi blocat - acesta
este scos din \textit{ready\_list} si adaugat in lista de asteptare a lacatului respectiv, lista fiind sortata dupa
prioritatea thread-urilor care asteapta. Deasemenea, in cazul in care prioritatea thread-ului blocat este mai mare ca 
cea a detinatorului lacatului, este necesara o donatie de prioritate. Aceasta donatie poate fi clasificata in doua cateogrii:
\begin{itemize}
 \item Detinatorul lacatului este READY - se face o donatie de la thread-ul curent la detinatorul lacatului si detinatorul 
 lacatului este repozitionat in \textit{ready\_list}.
 \item Detinatorul lacatului este BLOCKED si a facut o donatie de prioritate - este necesara o propagare a donatiilor. Se
 face donatie de la thread-ul curent la detinatorul lacatului si detinatorul lacatului este repozitionat in lista de asteptare
 a lacatului care l-a blocat. Apoi se efectueaza propagarea donatiei de prioritate.
\end{itemize}

La apelul functiei \textit{lock\_release}, daca thread-ul curent a primit donatie de prioritate, este necesara recalcularea 
prioritatii sale. In cazul in care niciun thread nu mai asteapta dupa thread-ul curent, acesta revine la prioritatea sa initiala. 
Altfel, noua sa prioritate va fi maximul dintre prioritatile thread-urilor care asteapta dupa thread-ul curent.

\section{Multilevel Feedback Queue Scheduler}
Acest planificator decide ce threaduri vor rula modificand dinamic prioritatile acestora pe baza a doua
caracteristici ale lor: cat de dragute sunt, \textit{niceness}, si cat de consumatoare de procesor sunt, \textit{recent\_cpu}.

\subsection{Structuri de date}
Pentru a realiza acest lucru am adaugat in structura de thread din \textit{"threads/thread.h"} doua campuri care
sa inglobeze cele doua caracteristici pe baza carora se recalculeaza prioritatile. 

\begin{lstlisting}
struct thread
  {
    /* Owned by thread.c. */
    tid_t tid;                          /* Thread identifier. */
    enum thread_status status;          /* Thread state. */
    char name[16];                      /* Name (for debugging purposes). */
    uint8_t *stack;                     /* Saved stack pointer. */
    int priority;                       /* Priority. */
    struct list_elem allelem;           /* List element for all threads list. */

		int64_t wakeup_time;
		int niceness;
		int recent_cpu;

    /* Shared between thread.c and synch.c. */
    struct list_elem elem;              /* List element. */

#ifdef USERPROG
    /* Owned by userprog/process.c. */
    uint32_t *pagedir;                  /* Page directory. */
#endif

    /* Owned by thread.c. */
    unsigned magic;                     /* Detects stack overflow. */
  };

\end{lstlisting}

De asemenea se va adauga o variabila globala, \textit{loag\_avg}.

\subsection{Algoritmi}
Presupunem ca threadurile A, B si C au niceness 0, 1 si 2. Fiecare au o valoare de \textit{recent\_cpu} de 0. Se 
completeze tabelul de mai jos aratand decizile de planificare, prioritatile si valorile de 
\textit{recent\_cpu} pentru fiecare thread pentru numarul de tickuri.

\begin{center}
 \begin{tabular}{||c c c c||} 
 \hline
 timer ticks & recent\_cpu  A B C & priority  A B C & thread to run \\ [0.5ex]
  & A B C & A B C & \\
 \hline\hline
 0 & 0 0 0& 63 61 59 & A \\ 
 \hline
 4 & 4 0 0& 62 61 59 & A \\
 \hline
 8 & 8 0 0& 61 61 59 & A \\
 \hline
 12 & 12 0 0& 60 61 59 & B \\
 \hline
 16 & 12 4 0& 60 60 59 & A \\
 \hline
 20 & 16 4 0& 59 60 59 & B \\
 \hline
 24 & 16 8 0& 59 59 59 & A \\
 \hline
 28 & 20 8 0& 58 59 59 & B \\
 \hline
 32 & 20 12 0& 58 58 59 & C \\
 \hline
 36 & 20 12 4& 58 58 58 & A \\
 \hline
\end{tabular}
\end{center}

Performanta acestui planificator o sa fie afectata deoarece de fiecare data cand trebuie sa se faca
schedule trebuie sa se recalculeze prioritatile. Acest lucru conduce la introducerea unor intarzieri.

Formulele pentru calcularea valorilor pentru \textit{priority}, \textit{niceness} si \textit{ready\_threads} presupun numere 
intregi, insa pentru calcularea valorilor \textit{recent\_cpu} si \textit{load\_avg} trebuie folosite numere reale. Pintosul 
nu pune la dispozitie suport pentru aritmetica in virgula mobila. Astfel calculele efectuate cu numere
reale trebuie simulate utilizand numere intregi. 

Astfel se va crea un header numit \textit{fixed\_point\_arithmetic.h} care va contine macrourile necesare
executarii operatiilor cu numere reale.

Fie x si y numere in fix point arithmetic, n un numar intreg, p + q = 31 si f fiind 1 << q. Astfel se 
vor implementa macrouri pentru urmatoarele operatii:
\begin{description}
\item[$\bullet$] conversia lui n la fix point: n * f
\item[$\bullet$] conversia lui x la intreg (rotunjire spre zero): x / f
\item[$\bullet$] conversia lui x la intreg (rotunjire spre cel mai apropiat) : (x + f / 2) / f daca x>=0 sau (x - f / 2) / f daca x<=0
\item[$\bullet$] suma dintre x si y: x + y
\item[$\bullet$] diferenta dintre x si y: x - y
\item[$\bullet$] suma dintre x si n: x + n * f
\item[$\bullet$] diferenta dintre x si n: x - n * f
\item[$\bullet$] inmultirea dintre x si y: ((int64\_t)x) * y / f
\item[$\bullet$] inmultirea dintre x si n: x * n
\item[$\bullet$] impartirea lui x la y: ((int64\_t)x) * f / y
\item[$\bullet$] imparitrea lui x la n: x / n
\end{description}

De asemenea trebuie ignorate apelurile la functia \textit{set\_priority} si implementate functiile 
\textit{thread\_set\_nice}, \textit{thread\_get\_nice}, \textit{get\_load\_avg}, \textit{get\_recent\_cpu}, \textit{increase\_recent\_cpu}, \textit{calc\_recent\_cpu}.

\end{document}
